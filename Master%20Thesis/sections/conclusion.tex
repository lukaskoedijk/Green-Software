\chapter{Conclusion}
\label{ch:conclusion}

\section{Conclusion}
(1) \textbf{Is there a difference in the energy consumption of software projects in different programming languages that have the same functionality}
\begin{itemize}
\item Hypothesis: There is no difference in the energy consumption of software projects in different programming languages that have the same functionality.
\item Alternative hypothesis: There is a difference in the energy consumption of software projects in different programming languages that have the same functionality.
\item Method: Using twice the one sided Mann Whitney U test where in both cases the null hypothesis is that the distributions are the same but the alternative hypothesis is that one distribution is greater or less.
\item Result: For \textit{node28} in the most cases, 294 of 306 cases, we can reject the null hypothesis, but in 12 cases we can not. This is roughly 4\% and in these cases we cannot reject the null hypothesis. Which does not mean that they are not different, maybe the test is not powerful enough. For \textit{node29} the numbers change a bit, there we cannot reject the null hypothesis in 30 of the 306 cases. Which is roughly 10\%.
\item Conclusion: For most cases we reject the null hypothesis and accept the alternative hypothesis. Thus for most cases there is a difference in the energy consumption of software projects in different programming languages that have the same functionality.\\
\end{itemize}


(2) \textbf{Is there a difference in energy consumption of different software projects (in the same programming language) that have the same functionality}
\begin{itemize}
\item Hypothesis: There is no difference in energy consumption of different software projects, in the same programming language, that have the same functionality.
\item Alternative hypothesis: There is a difference in energy consumption of different software projects, in the same programming language, that have the same functionality.
\item Method: Compare programs if they have the same language and solve the same problem. Perform twice the one sided Mann Whitney U test where in both cases the null hypothesis is that the distributions are the same but the alternative hypothesis is that one distribution is greater or less.
\item Result: We cannot reject the null hypothesis in 24 of 479 (roughly 5\%) cases for the \textit{node28} and 50 of 479 (roughly 10\%) cases for the \textit{node29}.
\item Conclusion: We for most cases reject the null hypothesis and accept the alternative hypothesis. Thus in most cases there is a difference in the energy consumption of software projects that are in the same programming language and solve the same problem.
\end{itemize}


\section{Future work}
\label{sec:future_work}
