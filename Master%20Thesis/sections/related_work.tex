\chapter{Related work}
\label{ch:related_work}
%We divide the related work into ... categories:  ...
%-overview
%-subsection{finding optimal energy consumption}
%-subsection{estimating energy consumption}
%-subsection{hardware influence on energy consumption}
%-subsection{software influence on energy consumption}
%-subsection{other}

To get an overview of the related research I made a table, table \ref{table:LiteratureStudy}, comparing the different papers. Most research is about reducing the energy consumption of a specific piece of hardware, for example scheduling on a multi-core processor. There are also some papers about reducing the energy consumption in the software and the decision making process. There is also some research done on the energy consumption of software, but their research goal was to estimate the energy consumption. Therefore what I miss and want to look into is to research if there is a good way of writing code regards the energy consumption.

\begin{table}[h]
\resizebox{\textwidth}{!}{%
\begin{tabular}{|l|l|l|l|}
\hline
\textbf{Papers}& \textbf{Type of Research} & \textbf{Unit of Analysis} & \textbf{Goal} \\ \hline
\cite{verdecchia2017estimating} & Controlled Experiment & \begin{tabular}[c]{@{}l@{}}Deployment strategies, \\ releases and use case scenarios\end{tabular} & Finding optimal energy consumption \\ \hline
\cite{filip2015bats} & \begin{tabular}[c]{@{}l@{}}Case Study \&\\ Controlled Experiment\end{tabular} & HPC bag of task applications & Finding optimal energy consumption \\ \hline
\cite{acar2016impact} & Case Study & Small functions & Estimating energy consumption \\ \hline
\cite{liqat2013energy} & \begin{tabular}[c]{@{}l@{}}Case Study \&\\ Controlled Experiment\end{tabular} & Small functions & Estimating energy consumption \\ \hline
\cite{jayaseelan2006estimating} & Case Study & Task on complex micro-architectures & Estimating worst-case energy consumption \\ \hline
\cite{sahin2014code} & Empirical Study & Six commonly used refactorings & \begin{tabular}[c]{@{}l@{}}Finding impact of refactorings\\ on energy consumption\end{tabular} \\ \hline
\cite{pereira2017energy} & Controlled experiment & Programming languages & \begin{tabular}[c]{@{}l@{}}Rank programming languages based on \\speed, memory usage and energy consumption\end{tabular} \\ \hline
\cite{zakarya2013energy} & Case Study & Multi-core processor scheduling & Efficient workload partitioning \\ \hline
\cite{zhu2004reducing} & Case Study & Cache storage management algorithms & Power aware cache management\\ \hline
\cite{manotas2014seeds} & Case Study & Java applications & \begin{tabular}[c]{@{}l@{}}Framework to automate decision-making\\ support regarding energy consumption\end{tabular} \\ \hline
\cite{mahmoud2013green} & Case Study & Software process & Two level green software model \\ \hline

\end{tabular}%
}
\caption{Overview of related research}
\label{table:LiteratureStudy}
\end{table}

\subsection{Estimating energy impact of software releases and deployment strategies: the KPMG case study \cite{verdecchia2017estimating}}
This paper looks at the impact of releases and deployment strategies of a software product on the energy consumption. They used a controlled experiment where the variables they changed where deployment strategies, releases and use case scenarios. The variables they measured during their tests were power consumption and execution time. They saw that both the releases and deployment strategies impacted the energy consumption and that this impact was influenced by which use case scenario they used. Therefore they concluded that there is no absolute optimal option for releases and deployment strategies with respect to energy consumption. They also found that the execution time had a bigger impact on the energy consumption than the power consumption, because of the low variability in power consumption.

\subsection{The Impact of Source Code in Software on Power Consumption \cite{acar2016impact}}
This paper looks at different techniques to measure the power consumption. Then they propose a model to measure the power consumption and they used this model in their implementation named \textit{Tool to Estimate Energy Consumption} (TEEC). They test their implementation against a power meter, but they do not mention how accurate their implementation is. The figure they use at the validation is also not clear, they just state that it shows the same behaviour as TEEC. They find that the power consumption of unoptimized code is higher and has a longer execution time than the optimized code. They do not mention it but looking at their graphs, the unoptimized and optimized code seem to have the same peaks only on different time steps for the optimized code is faster.

\subsection{Energy Consumption Analysis of Programs based on XMOS ISA-Level Models \cite{liqat2013energy}}
This paper estimates the energy consumption by developing a model which can be applied at instruction set simulation level. This was done by designing a translation from instruction set architecture code to Horn- clause representation and this model is called in the paper \textit{Instruction Set Simulation} (ISS). They also use the CiaoPP general resource analysis framework, which is low level, to model the energy consumption. They named it \textit{Static Resource Analysis} (SRA) in the paper. In their experiments they only use small functions to test and the results were compared to a mathematical equation. The found that the ISS function is less accurate when the value of N increases and that the SRA function is only not accurate for small value of N. Here N is the input value of the function that is tested for its energy consumption.

\subsection{E-BATS: ENERGY-AWARE SCHEDULING FOR BAG-OF-TASK APPLICATIONS IN HPC CLUSTERS \cite{filip2015bats}}
This paper looks at the scheduling of bags of task application in High performance Computing (HPC). They delved into the trade-off between energy consumption and performance by finding a optimal point for both variables. This was calculated by designing a dynamic Hill Climbing algorithm. Their algorithm uses less then 12\% of the resources an exhausted search would use to find a majority of points close to the optimal for the trade-off in 10 of the 12 scenarios. They validated their algorithm by implementing it and running a wide range of HPC bag of task applications. They found that the estimations of their algorithm have an error below 5\%.

\subsection{SEEDS: A Software Engineer’s Energy-Optimization Decision Support Framework \cite{manotas2014seeds}}
This paper implements a framework that automatically optimizes the energy consumption of a Java software project. The framework does this by running multiple different given options and testing which option consumes the least amount of energy. Thus as input the framework needs a list of possible changes. Because the framework needs possible changes we don't know if the resulting code is the most energy efficient version, only that it is more energy efficient then the input. They showed that by letting their framework chose which library to use they reduced their energy consumption by 17\%.

%Mann-Whitney-Wilcoxon to see if statistically significant difference
\subsection{How Do Code Refactorings Affect Energy Usage? \cite{sahin2014code}}
This paper addresses the lack of information about the energy impact of code refactorings. They did this by testing the energy impact of 197 projects when the using six commonly used refactorings. From this test they found that refactorings can influence the energy consumption. Also they find that one refactoring does not necessarily have the same influence on the energy consumption when used with different projects.


\subsection{Reducing Energy Consumption of Disk Storage Using Power-Aware Cache Management \cite{zhu2004reducing}}
This paper tries multiple algorithms for storage cache management to decrease the energy consumption. One algorithm is an offline greedy algorithm and they also propose an online algorithm. They evaluate their algorithms by simulating a complete storage system, enhancing the DiskSim simulator. Their greedy algorithm results in 16\% less energy used then the LRU algorithm. They also find that the cache policy write-back can use 20\% less energy then write-through.



\subsection{A Green Model for Sustainable Software Engineering \cite{mahmoud2013green}}
This paper makes a two level green software model. The first level is about making the software process more energy efficient. This new process is a hybrid of the sequential, iterative, and agile development processes. The second level is about the role software tools can have on improving the energy efficiency of software. They also discus the relationship between the two levels.


\subsection{Estimating the Worst-Case Energy Consumption of Embedded Software \cite{jayaseelan2006estimating}}
This paper estimates the worst-case energy consumption of a task on complex micro-architectures. This is important for battery-operated embedded devices, where we don't what the battery to drain empty before a critical task is completed. They test their result against a commonly used benchmark and they find that their values have at most 33\% difference with the benchmark.

\subsection{Energy Efficient Workload Balancing Algorithm for Real-Time Tasks over Multi-Core \cite{zakarya2013energy}}
This paper proposes an algorithm to makes sure all cores in a multi-core processor have the same workload. This is reducing energy consumption because multiple single core processors consume more energy.

\subsection{Energy Efficiency across Programming Languages
 \cite{pereira2017energy}}
 This paper tries to find a connection between the speed, memory usage and energy consumption of a programming language. They do this by choosing the fastest implementation of the exact same algorithm, defined in the computer language benchmarks game, in different programming languages. From these programs they measured the execution time, memory usage and energy consumption. They used Intel’s Running Average Power Limit (RAPL) tool to measure the energy consumption and for the memory usage and execution speed they used the \textit{time} command available in Unix-based systems. They find that a faster programming language does not necessarily have a lower energy consumption and memory usage does not relate to energy consumption. A big problem with this paper is that in their threads to validity paragraph they defend their implementation instead of stating what could be wrong with their implementation. My main problem with this paper is that they compare languages based on the fastest solution in some competition and these are not comparable. Because there can be languages where the fastest solution given in the competition is not the fastest at all.
 
 %  negative correlation factor?
%  Evaluation on the same application as used for calculating this new equation
 \subsection{Accurately Simulating Energy Consumption of I/O-intensive Scientific Workflows \cite{ferreiradasilva-iccs-2019}}
 This paper looks at two applications that are I/O heavy. Different tasks were run for these applications and the energy consumption was measured. Another variable in their experiments was the amount of cores used. They compared the energy values measured with a commonly used estimation scheme for the energy consumption which only looks at the CPU utilization. They noticed a difference and the correlation between power consumption and CPU utilization was close to zero. The reason for this was that the estimation scheme didn't factor in the energy consumption of I/O operations. Therefor they came up with a scheme that factors in the CPU utilization and the I/O operations. This scheme used values from the tests to put in different values in the formula and was tested against the two applications and they found a small error. This is an issue for me, because you expect the data you used to create a model to fit the model. A better approach would have been to use one application for calculating the values and the other for the validation.
 
