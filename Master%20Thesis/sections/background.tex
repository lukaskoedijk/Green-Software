\chapter{Background}
\label{ch:background}
% This chapter will present the necessary background information for this thesis. First, we define some basic terminology that will be used throughout this thesis. Next, ...

\section{Statistics}
Not understanding the terminology of statistics may lead to confusion, therefor here are some basic principles in statistics. When performing a statistical test we have a null hypothesis and an alternative hypothesis. Every test has it specific null and alternative hypothesises and the goal of the test is to reject the null hypothesis. Rejecting the null hypothesis is done by looking at the resulting \textit{p}-value of the test. The \textit{p}-value is the chance that the value of the statistical test occurs if the null hypothesis is true on a zero to one scale. We therefor reject the null hypothesis if we think this chance in too low, thus below a certain threshold. This threshold is determined beforehand and is called the \textit{alpha}-value, a common value for  \textit{alpha} is 5\%. When we cannot reject the null hypothesis, thus the \textit{p}-value is not below $0.05$, it does not necessarily mean that the null hypothesis is true. It could be the case that the test is not powerful enough or your data size is too small.


%compiler
%miss framewrok benchmark game
%energy uitleggen
%miss units uitleggen
%basic statistics: null hypothese, alpha, p, rejecting h0 more valuable