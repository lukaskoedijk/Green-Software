\chapter{Data}
\label{ch:data}
In this chapter we show the approach followed when selecting the data used for testing. We first discus which programming languages we should investigate. After that we discus what the source is for our programs and which issues we came across when using those programs. 

\section{Programming languages}
To find data we first need to decide which programming languages to choose for testing. We did this by looking into what the most popular and most used programming languages are. For this question it depends who you ask which result you get. Therefor four sources were used to determine which programming language to use. The four sources are git-hub, indeed, TIOBE and PYPL.\\
%what the resulting most popular programming language is

When languages are in all the top tens of the four sources they were labelled as popular and these are the languages included in this research, thus the languages Java, JavaScript, Python, C\#, C++ and PHP. We also choose to investigate C and Ruby. C because it was in three of the four top tens and it is well-established and known for a high level of control over available resources, thus having a potential impact on the energy consumption. Ruby was in two of the top tens, but also 13th according to TIOBE and 12th in the PYPL index. Thus Ruby was close to be in all the top tens and therefor also chosen to be investigated.


% Binarytrees:
% -Java:
% 	-1 -> not correct output
% 	steps:
% 		- mv binarytrees.java-7.java binarytrees.java
% 		- javac -d . binarytrees.java
% 		- java binarytrees 21
% 		- rm -v binarytrees.class binarytrees\$1.class binarytrees\$TreeGenerator.class binarytrees\$TreeNode.class 		(just for clean up, but new javac overwrites these files)
%   das5:
%       -srun javac binarytrees.java && java binarytrees 21
%       -srun -N1 -w "node028" run_continues.sh  -o "test_java" -p "3" -f "javac -d /home/lkoedijk/binarytrees /home/lkoedijk/binarytrees/binarytrees.java,java -cp /home/lkoedijk/binarytrees/ binarytrees 21"
%
% 	java version local (java -version):
% 		- java version "1.8.0_191"
% 		- Java(TM) SE run time Environment (build 1.8.0_191-b12)
% 		- Java HotSpot(TM) 64-Bit Server VM (build 25.191-b12, mixed mode)
%     das5:
%         -openjdk version "1.8.0_161"
%         -OpenJDK run time Environment (build 1.8.0_161-b14)
%         -OpenJDK 64-Bit Server VM (build 25.161-b14, mixed mode)

% -Python:
%     steps:
%         -mv binarytrees.python3.py binarytrees.py
%         -python3 binarytrees.py 21
%     das5:
%         -srun -N1 -w "node028" run_continues.sh  -o "test_py" -p "3" -f "python3 /home/lkoedijk/binarytrees/binarytrees.python3.py 21"
        
%     python3 local version (python3 --version):
%         - Python 3.6.0 :: Anaconda custom (x86_64)
%     das5:
%         -Python 3.4.5
        
% -JavaScript:
%     steps:
%         -cp -L binarytrees.node binarytrees.js
%         -node binarytrees.js 21
%    das5:
%       -srun -N1 -w "node028" run_continues.sh  -o "test_javascript" -p "3" -f "node /home/lkoedijk/binarytrees/binarytrees.js 21"
        
%     javascript local version (node -v):
%         -v11.10.0
%     das5:
%         -v6.12.3
        
% -PHP:
%     -4, 5, 6 -> error: undefined function pcntl_fork()
%     steps:
%         -php -n -d memory_limit=4096M binarytrees.php 21
%     das5:
%         -srun -N1 -w "node028" run_continues.sh  -o "test_php" -p "3" -f "php -n -d memory_limit=4096M /home/lkoedijk/binarytrees/binarytrees.php 21" 
        
%     php local version (php -v):
%         -PHP 7.1.23 (cli) (built: Nov  7 2018 18:20:35) ( NTS )
%         -Copyright (c) 1997-2018 The PHP Group
%         -Zend Engine v3.1.0, Copyright (c) 1998-2018 Zend Technologies
%     das5:
%         -PHP 5.4.16 (cli) (built: Mar  7 2018 13:34:47) 
%         -Copyright (c) 1997-2013 The PHP Group
%         -Zend Engine v2.4.0, Copyright (c) 1998-2013 Zend Technologies
        
% -Ruby:
%     steps:
%         -ruby -W0 binarytrees.yarv 21
%     das5:
%         -srun -N1 -w "node028" run_continues.sh  -o "test_php" -p "3" -f "ruby -W0 /home/lkoedijk/binarytrees/binarytrees.yarv 21"
        
%     ruby local version (irb -> RUBY_VERSION -> exit):
%         -2.3.7
%     das5:
%         -2.0.0

% -C:
%     -3 -> error apr_pools.h not found
%     steps:
%         -/usr/bin/gcc -pipe -Wall -O3 -fomit-frame-pointer -march=native -I/usr/include/apr-1.0 binarytrees.gcc.c -o binarytrees.gcc_run -lapr-1
%         -./binarytrees.gcc_run 21
    % das5:
    %     -(for program 5)gcc -pipe -Wall -O3 -fomit-frame-pointer -march=native -I/usr/include/apr-1.0 -pthread binarytrees.gcc-5.c -o binarytrees.gcc_run
    %     -(for program 0)gcc -pipe -Wall -O3 -fomit-frame-pointer -march=native -I/usr/include/apr-1.0 /home/lkoedijk/binarytrees/binarytrees.gcc.c -o binarytrees.gcc_run -lm
    %     -(for program 3) need apr_pools.h
    %     -srun -N1 -w "node028" run_continues.sh  -o "test_c" -p "3" -f "/home/lkoedijk/binarytrees/binarytrees.gcc_run 21"
        
%     c local version (gcc -v):
%         -Configured with: --prefix=/Applications/Xcode.app/Contents/Developer/usr --with-gxx-include-dir=/Applications/Xcode.app/Contents/Developer/Platforms/MacOSX.platform/Developer/SDKs/MacOSX10.14.sdk/usr/include/c++/4.2.1
%         -Apple LLVM version 10.0.0 (clang-1000.11.45.5)
%         -Target: x86_64-apple-darwin18.2.0
%         -Thread model: posix
%         -InstalledDir: /Applications/Xcode.app/Contents/Developer/Toolchains/XcodeDefault.xctoolchain/usr/bin
%     das5:
%         -Using built-in specs.
%         -COLLECT_GCC=gcc
%         -COLLECT_LTO_WRAPPER=/cm/local/apps/gcc/6.3.0/libexec/gcc/x86_64-pc-linux-gnu/6.3.0/lto-wrapper
%         -Target: x86_64-pc-linux-gnu
%         -Configured with: ../gcc-6.3.0/configure --prefix=/cm/local/apps/gcc/6.3.0 --enable-languages=c,c++,fortran --with-gmp-include=/root/rpmbuild/BUILD/gcc-6.3.0-obj/../gcc-6.3.0/our-gmp --with-gmp-lib=/root/rpmbuild/BUILD/gcc-6.3.0-obj/../gcc-6.3.0/our-gmp/.libs --with-mpc-include=/root/rpmbuild/BUILD/gcc-6.3.0-obj/../gcc-6.3.0/our-mpc/src --with-mpc-lib=/root/rpmbuild/BUILD/gcc-6.3.0-obj/../gcc-6.3.0/our-mpc/src/.libs --with-mpfr-include=/root/rpmbuild/BUILD/gcc-6.3.0-obj/../gcc-6.3.0/our-mpfr/src --with-mpfr-lib=/root/rpmbuild/BUILD/gcc-6.3.0-obj/../gcc-6.3.0/our-mpfr/src/.libs
%         -Thread model: posix
%         -gcc version 6.3.0 (GCC) 
        
% c++:
%     -2,6 -> wrong answers
%     -9 -> missing apr_pools.h
%     steps:
%         -(for 1,3)g++ -c -pipe -O3 -fomit-frame-pointer -march=native /home/lkoedijk/binarytrees/binarytrees.gpp-3.c++ -o /home/lkoedijk/binarytrees/binarytrees.gpp.c++.o && g++ /home/lkoedijk/binarytrees/binarytrees.gpp.c++.o -o /home/lkoedijk/binarytrees/binarytrees.gpp_run -lpthread -lboost_system
%         -(for 8)g++ -c -pipe -O3 -fomit-frame-pointer -march=native -fopenmp /home/lkoedijk/binarytrees/binarytrees.gpp-8.c++ -o /home/lkoedijk/binarytrees/binarytrees.gpp.c++.o && g++ /home/lkoedijk/binarytrees/binarytrees.gpp.c++.o -o /home/lkoedijk/binarytrees/binarytrees.gpp_run -lboost_system -fopenmp
%         -./binarytrees.gpp_run 21

    % fannkuchredux:
    %     -2,3,5 error
    
%     c++ local version (g++ -v):
%         -Configured with: --prefix=/Applications/Xcode.app/Contents/Developer/usr --with-gxx-include-dir=/Applications/Xcode.app/Contents/Developer/Platforms/MacOSX.platform/Developer/SDKs/MacOSX10.14.sdk/usr/include/c++/4.2.1
%         -Apple LLVM version 10.0.0 (clang-1000.11.45.5)
%         -Target: x86_64-apple-darwin18.2.0
%         -Thread model: posix
%         -InstalledDir: /Applications/Xcode.app/Contents/Developer/Toolchains/XcodeDefault.xctoolchain/usr/bin
%     das5:
%         -Using built-in specs.
%         -COLLECT_GCC=g++
%         -COLLECT_LTO_WRAPPER=/cm/local/apps/gcc/6.3.0/libexec/gcc/x86_64-pc-linux-gnu/6.3.0/lto-wrapper
%         -Target: x86_64-pc-linux-gnu
%         -Configured with: ../gcc-6.3.0/configure --prefix=/cm/local/apps/gcc/6.3.0 --enable-languages=c,c++,fortran --with-gmp-include=/root/rpmbuild/BUILD/gcc-6.3.0-obj/../gcc-6.3.0/our-gmp --with-gmp-lib=/root/rpmbuild/BUILD/gcc-6.3.0-obj/../gcc-6.3.0/our-gmp/.libs --with-mpc-include=/root/rpmbuild/BUILD/gcc-6.3.0-obj/../gcc-6.3.0/our-mpc/src --with-mpc-lib=/root/rpmbuild/BUILD/gcc-6.3.0-obj/../gcc-6.3.0/our-mpc/src/.libs --with-mpfr-include=/root/rpmbuild/BUILD/gcc-6.3.0-obj/../gcc-6.3.0/our-mpfr/src --with-mpfr-lib=/root/rpmbuild/BUILD/gcc-6.3.0-obj/../gcc-6.3.0/our-mpfr/src/.libs
%         -Thread model: posix
%         -gcc version 6.3.0 (GCC) 

%C#:
%   steps:
%       -mcs binarytrees.cs
%       -srun -N1 -w "node028" run_continues.sh  -o "test_c" -p "3" -f "mono /home/lkoedijk/binarytrees/binarytrees.exe 21"


%!!!!Knucleotide error with only javascript file, different error local and on das
%!!!!Knucleotide error with only c file, no khash.h file
%!!!!Pidigits all three python files, No module named 'gmpy'
%!!!!Regexredux php with error, Call to undefined function msg_get_queue()
%!!!!Regexredux c++ 1,2,5: fatal error: 're2/re2.h' file not found
%               c++ 4:fatal error: string_view: No such file or directory

\section{Gathering data}
To be able to compare different programs they need to have the same functionality. A source for programs that have the same functionality is  \textit{the computer language benchmark game} \cite{gouy:2019}. All the programs for the ten problems in our seven programming languages were downloaded. These programs were then tested to see if they could compile, run and result in the correct output. This was all done on the DAS5 to make sure there were no local dependencies. All the programs that did not compile, or resulted in run time errors or incorrect output were excluded from the data set. There were three problems, Knucleotide, Pidigits and Regexredux, that for different programming languages use a partly different algorithm because of different library implementations. Therefor we excluded these problems from the data set. At the time of conducting the research, no working JavaScript implementations of the Mandelbrot problem were available. Nonetheless this problem will still be used and we will just have an empty spot in our results.\\

% that didn't had for every programming language a program that could be run or had the correct output and they also had . I decided to still run these problems and just have a few empty slot in our results. For Knucleotide the programming languages were JavaScript and C, for Regexredux it was PHP and C++ and for Pidigits it was Python3. I have this data set on a public Git-Hub repository along side other material needed for my thesis. The link to this repository is \url{https://github.com/lukaskoedijk/Green-Software} and for the data set move to \textit{EnergyMeasurement/programs}.\\

Before running all the programs some needed to be compiled first. The compilation step of languages that have a separated compilation step were not included in the energy measuring. The reason for this is that a finished program is compiled once and then could be used multiple times. This does mean that the languages JavaScript, Python, PHP and Ruby have a bit of a disadvantage, because they don't have a separate compilation step. The compiler can nowadays do a lot of optimizations of the code. During this research we want to see the result of user decisions on the energy consumption. Therefore besides testing the programs with the compilation flags used on the language benchmark game we also tested with as few flags as possible. This means that we removed all the optimization flags except the ones needed for the compilation. Also the compiler version of the different languages is important. To give a good pictures of were we stand today we need to use the most recent stable version and also the most commonly used one. Unfortunately it isn't that easy to update the language versions on the DAS. Therefore I used the versions that were already on their and these programming language versions are listed alongside the compiler used in table \ref{tab:version}.

\begin{table}[h]
\centering
\begin{tabular}{|l|l|l|}
\hline
\textbf{Languages}  & \textbf{Compiler} & \textbf{Version}   \\ \hline
Java       & javac    & 1.8.0\textunderscore161 \\ \hline
JavaScript & node     & 6.12.3    \\ \hline
Python     & python   & 3.4.5     \\ \hline
PHP        & php      & 5.4.16    \\ \hline
C          & gcc      & 6.3.0     \\ \hline
C++        & gcc      & 6.3.0     \\ \hline
C\#         & mcs      & 5.10.1.20 \\ \hline
Ruby       & ruby     & 2.0.0     \\ \hline
\end{tabular}
\caption{All the different compilers and versions of the programming languages used.}
\label{tab:version}
\end{table}

This all resulted in a large data set of 202 programs with 67 programs that had to run twice, once with and once without flags. Running all these programs takes about eight hours to complete. Unfortunately we were only allowed to use a node on the DAS for 30 minutes to run one script during working hours. This means that we had to run it at night and in the weekend, which limited the amount of data points that were measured. Of course running programs more often would give more accurate results, but this was just too time consuming.\\

Some programs were too fast to get a good amount of energy measurement points. To solve this problem those programs were run for multiple times during one measurement and then afterwards their energy consumption was divided by the amount of times the programs was run. Another problem that occurred was that retrieving a measure point takes a variable amount of time. This caused some programs to still not have a good amount of energy measurement points. Therefor I decided to set a minimum amount of 30 of measurement points needed in order for it to be used. \\
% With data from the benchmark game I calculated that to run all the programs for the eight programming languages for the ten problems would take around nine hours. Because this is a large amount of time I decided to only measure every program three times, which would take around 27 hours. Of course running programs more often would give more accurate results, but this was just too time consuming.\\

% PHP:-d memory_limit=1024M -> set entry memory_limit to 1024M
% C:  -pipe -> use pipe instead of temporary files for communication                     between compiling stages 
%     -Wall -> turn on multiple warning flags
%     -O3 -> turns on a lot of optimization flags
%     -fomit-frame-pointer -> optimization, Omit the frame pointer in                            functions that don’t need one
%     -march=native -> pick the architecture of the host system
%     -lm -> link math library
%     (-lapr-1 -> link apr-1 library)
%     (-lgomp -> link gomp library)
%     -fopenmp -> enable handling of OpenMP directives(multithreading)
%     -D_FILE_OFFSET_BITS=64 -> this forces all file access calls to use                             the 64 bit variants
%     -I/usr/include/apr-1.0 -> specify directory to search for header                               files
%     -pthread -> define additional macros required for using the POSIX              threads library
%     -IInclude -> ?
%     -mno-fma -> enable the use of instructions in the no-fma instruction              set
%     -fno-finite-math-only -> ?
%     -lgmp -> link gmp library
%     -lpcre -> link pcre library
% c++:
%     -c -> compile or assemble the source files, but do not link
%     -lpthread -> link pthread library
%     -std=c++11(c++14) -> determine the language standard
%     -lboost_thread -> link boost_thread library
%     -lboost_system -> link boost_system library
%     -mfpmath=sse -> generate floating-point arithmetic for sse
%     -msse3 -> enable the use of instructions in the MSSE3 instruction              set
%     -Wl,--no-as-needed -> pass option no as needed to linker?
%     -lgmpxx -> link gmpxx library
%     -flto -> runs the standard link-time optimizer
%     -mtune=native -> specify the name of the target processor for which                     GCC should tune the performance of the code
%     -O0 -> disable optimizations
% Ruby: -w0 -> verbose mode silence

% In the previous chapter we found that the energy consumption of a program is the sum of the CPU, memory and disk energy consumption \cite{acar2016impact}. To get a good view on what influences the energy consumption, our problems need to be diverse when it comes to these three categories. The problems that we looked into are called Binarytrees, Fannkuchredux, Fasta, Mandelbrot, Nbody, Revcomp and Spectralnorm. For the Binarytrees problem a lot of trees are allocated and deallocated in memory and thus this is a memory intensive task. The fannkuchredux problem does a lot of calculations on all permutations and is thus CPU intensive. The Fasta problem creates and saves a large DNA sequence and is memory intensive. For the Mandelbrot problem a large bitmap is saved and thus is it memory intensive. The Nbody problem models the orbit of Jovian planets and is CPU intensive. The Revcomp problem reads a DNA sequences line by line, transforms them and writes the result to output. Therefor the Revcomp problem is disk intensive. For the Spectralnorm problem a lot of calculations are done on a large matrix and is thus CPU intensive. A more extensive explanation of the problems can be found on the computer language benchmark game website \cite{gouy:2019}. An overview of category and problem is shown in table \ref{tab:problems}.

% \begin{table}[h]
% \centering
% \begin{tabular}{|l|l|}
% \hline
% \textbf{Category} & \textbf{Problems}                           \\ \hline
% CPU      & Fannkuchredux, Nbody, Spectralnorm \\ \hline
% Memory   & Binarytrees, Fasta, Mandelbrot     \\ \hline
% Disk     & Revcomp                            \\ \hline
% \end{tabular}
% \caption{The job intensive categories the different problems are in.}
% \label{tab:problems}
% \end{table}

% tell which problem is I/O(disk?), memory or CPU insensitive
% description of problems can be found on site
% spectral-norm(maybe memory), n-body(maybe cpu)
% cpu(mainly calculations): fannkuchredux
% memory(lot of data stored or large data stored):mandelbrot, fasta, rev-comp, binarytrees (allocate/deallocate)
% disk: fasta, rev-comp

% pidigits*(maybe memory), regex-redux*(memory maybe disk), knucleotide*(memory and disk)

% minimal run time of 7 seconds to get a good amount of datapoints, 0.6-0.7 seconds per measure thus like 10 datapoint minimal, the more the better but we have not that much time

%-talk about amount of files and run time of single run of all programs
%-talk about run time (7 seconds)
%-talk about the reason programs/problems do not work and are leftout
%-tell that the versions are this version because these were on the das
%-talk about different types of problems (cpu, memory, disk)
%-talk about compilation (flags, pre-compilation), why not measured
%TODO:
