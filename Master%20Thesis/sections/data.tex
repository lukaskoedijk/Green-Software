\chapter{Data}
\label{ch:data}

\section{Gathering data}
To find data I first need to decide which programming languages to choose for testing. I did this by looking into what the most popular and most used programming languages are. For this question it depends who you ask what the resulting most popular programming language is. Therefor I used four sources to determine which programming language to use. The four sources are indeed, git-hub, TIOBE and PYPL.\\

Indeed is a job search site. They looked at the percentage of jobs with a programming language in their name in the tech software category \cite{ray:2018}. Thus the more job offering for a programming language the more popular that programming language is. The problem with this method is that job offerings do not show how many people work with a programming language, but only which programming language has a shortage in programmers. Based on data form September 2018 the top ten according to this method is Java, JavaScript, HTML, Python, C\#, C++, XML, Ruby, PHP and Perl.\\

Git-hub is a version control system where multiple programmers collaborate in a project. They looked at the amount of pull requests made for that language \cite{zap:2019}. The thought was that the language that programmers work a lot with on git is the most popular, but this is based on only the public repositories. In the first quarter of 2019 the top ten according to this method is JavaScript, Python, Java, Go, C++, Ruby, PHP, TypeScript, C\# and C. \\

TIOBE is a software quality company. They looked at the amount of hits they got when searching "[Language] programming" on a lot of different search engines \cite{tio:2019}. There are rules which a search engine needs to comply with for it to be used in the calculation and they also look a what type of hit they find to determine whether or not to use it in the calculations. The pitfall of this method is the favouritism for complex languages. When a language is more difficult to understand, more page of tutorials are needed and more questions about this language will be asked. As of April 2019 the top 10 according to this method is Java, C, C++, Python, Visual Basic .NET, C\#, JavaScript, SQL, PHP and Assembly language.\\

PYPL index stands for the PopularitY Programming Language index. They look at how many times a language tutorial is searched \cite{car:2019}. This method also has favouritism for complex languages, where programmers need to use the tutorials a lot because of the difficulty. Based on data form April 2019 the top ten according to this method is Python, Java, JavaScript, C\#, PHP, C/C++, R, Objective-C, Swift and Matlab.\\

When languages are in all the four top tens I labelled them as popular and these language I am going to investigate, thus the languages Java, JavaScript, Python, C\#, C++ and PHP. I also choose to investigate C and Ruby. C because it was in three of the four top tens and I found it interesting to see the difference in the variations of C like C++ and C\#. Ruby was in two of the top tens, but also 13th according to TIOBE and 12th in the PYPL index. Thus Ruby was close to be in all the top tens.


\section{Testing functionality}

