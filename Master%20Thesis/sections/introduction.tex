\chapter{Introduction}
\label{ch:introduction}
% Context: what is the bigger scope of the problem you are trying to solve? Try to connect to societal/economical challenges.
% Problem Analysis: Here you present your analysis of the problem situation that your research will address.
% How does this problem manifest itself at your host organisation?
% Also summarises existing scientific insight into the problem.
%maybe not only milieu but also money
Currently more and more people are concerned with global warming. Global warming is partly the result of the emission of greenhouse gasses during energy generation of conventional energy options. Because of this a lot of people want to change to green energy generation to help solve this problem. Another solution is to decrease the energy consumption, which is not only good for the environment but can also save a lot of money on the energy bill. The energy consumption of communication networks, personal computers and data centers world wide is increasing over the years \cite{van2014trends}. This happens at a growth rate of $10\%$, $5\%$ and $4\%$ respectively \cite{van2014trends}. Therefore it is important to research ways of decreasing the energy consumption. In the field of hardware there is, according to Koomey's law, an increase of the number of computations per Joule. However this is not enough, because tasks need more computations to complete due to the confidence in the improvement of hardware \cite{verdecchia2017estimating}. For this reason we need to look at possibilities in decreasing the energy consumption from a software perspective.


\section{Problem statement}
The energy consumption of communication networks, personal computers and data centers are increasing yearly. There are two ways of decreasing the energy consumption, by decreasing the energy consumption of hardware or software. Scientific research is mostly focused on decreasing the energy consumption of hardware. There are also some papers about reducing the energy consumption in the software process and the decision making process. A small bit of research is done on the energy consumption of software, but their research goal is to estimate the energy consumption. This thesis aims to investigate good ways of writing code with regards to energy consumption.

\subsection{Research questions}
%To tackle these issues, we investigate ...
During this research we answer the following two existence questions: (1) \textbf{Is there a difference in the energy consumption of software projects in different programming languages that have the same functionality} and (2) \textbf{Is there a difference in energy consumption of different software projects (using the same programming language) that have the same functionality}? To answer these questions we first need to answer the three description and classification questions listed below.
\begin{itemize}
    \item How can the energy consumption of a software project be measured?
    \item How do we prove that two programs have the same functionality?
    \item When is a difference in energy consumption significant?\\
\end{itemize}

Based on the results we find when answering the second existence question, we looked into what this difference is on code level. To do this we need to answer the descriptive/comparative question \textbf{what is the difference on code level between software project that are in the same programming language and have the same functionality, but have a difference in the energy consumption}? This question is only useful when there is a difference in the energy consumption of software projects in the same programming language that have the same functionality. \\
%Otherwise we can't link the difference in energy consumption to the difference in the code.\\

For the first research question the hypothesis is that there is a difference in the energy consumption based on the programming language chosen. The hypothesis of the second research question is that there is a difference in the energy consumption of different software projects (in the same programming language) that have the same functionality.

\subsection{Research method}
As method we use a controlled experiment. In this experiment we will run different software projects to measure the energy consumption. Here the projects are the variable input, the energy consumption the output and environmental variable such as the compiler used and energy consumption calculations should be constant.

\subsubsection{Data}
For the first research question we need programs that have the same functionality, but are written in a different programming language. Possible resources for such programs are library code, interview code, student assignments and code from competitions solving math problems. We first looked at library code, but found that libraries most of the time don't use the same algorithm. This is a problem because the research question does not target comparing algorithms, but the way a programmer writes code. Therefor we used the \textit{the computer language benchmark game} \cite{gouy:2019} as a source for programs that have the same functionality and use the same algorithm for solving the problem at hand. The second research question requires programs that offer the same functionality and are written in the same programming language. Here we could also use \textit{the computer language benchmark game} as most programs are available in different programming languages.

When proving that all the programs have the same functionality we could have defined properties for the input and output and test if these properties hold for all the programs \cite{mens2004survey}, but \textit{the computer language benchmark game} already makes sure that the programs offer the same functionality. This research builds upon these validations.

To prove that the programs consume different amounts of energy we use the one sided Mann Whitney U test. This test has as its hypothesis that two distributions are from the same population.

\subsubsection{Measuring energy consumption}
When measuring the energy consumption of a project you have to take into account the energy consumption of CPU, memory and disk \cite{acar2016impact}. The measurements can be done with a hardware or software approach. A hardware approach is more accurate but also more expensive \cite{acar2016impact}. This research uses a hardware approach and utilizes the DAS-4 and DAS-5 systems. The DAS is a distributed system that can also do energy measurements \cite{bal2016medium}. The nodes that can measure the energy are located at the VU cluster. To measure the energy we need to run a job on the DAS-5 and use the DAS-4 to measure the energy consumption. When running a job we need to specify which node we want to run the job on, because not all nodes can be measured for the energy consumption. 
%This is done by using the \textit{-w} flag while using the command \textit{srun}. The measuring can then be done by using the \textit{smnpwalk} command and then filtering out the information by using \textit{fgrep Energy}.

\subsubsection{Code level}
When there is a difference in the energy consumption of different software projects in the same programming language we will look at the different projects on code-level. Here we will try to find what is causing a project to have a lower or higher energy consumption. There are too many combinations of programs and programming languages to look at, thus we need to make a selection. We choose to look at the programming languages that had the biggest range in energy consumption for all the problems, these languages are Python, Ruby and PHP. For every problem we looked to the two programs that differ the most for those three programming languages. Unfortunately not all the problems had enough implementations and some didn't have implementations where we could prove their energy consumption was different. We will look at the program combinations side by side and write down the differences that are observed. Then we analyze these differences and determine if some are occurring more often then others. This could get us a preliminary understanding of what causes the difference in energy consumption of different programs, but to really find out what causes the difference extended research should be done. We will perform a small scale example of such extended research by looking at two differences and implementing them in the well known nth prime number problem. These programs will each run 30 times and then we test if there is a difference and which one is performing better regards the energy consumption using the Mann Whitney U test.

%To get an even better understanding we should measure the energy consumption of only small changes.
%These findings will then be tested by writing our own two versions, where before testing we think one is written badly regards the energy consumption and one that is written good. The problem these two programs will solve is the well known problem of calculating the nth Fibonacci number. We then run the two version and check if the good version has indeed a lower energy consumption using the Mann Whitney U test. 

\section{Contributions}
Our research makes the following contributions:
\begin{enumerate}
	\item An energy consumption measurement set-up
	\item A data set of software programs featuring the same functionality
	\item Comparison of programming languages on energy consumption efficiency
	\item Preliminary findings for writing good software regards the energy consumption
\end{enumerate}

\section{Outline}
% In Chapter~\ref{ch:background} we describe the background of this thesis. 
% Chapter~\ref{ch:research} describes ... 
% Results are shown in Chapter~\ref{ch:results} and discussed in Chapter~\ref{ch:discussion}. Chapter~\ref{ch:related_work}, contains the work related to this thesis.
% Finally, we present our concluding remarks in Chapter~\ref{ch:conclusion} together with future work.
In chapter \ref{ch:background} we describe background knowledge that is needed to understand the rest of the thesis. Chapter \ref{ch:energy_measurement} explains the energy measurement set-up we used and in chapter \ref{ch:data} we discuss the programs we choose to investigate. The results are shown in chapter \ref{ch:results} and in chapter \ref{ch:discussion} we discuss these results. In chapter \ref{ch:related_work} we discus papers related to our subject and lastly we conclude our research in chapter \ref{ch:conclusion}.

