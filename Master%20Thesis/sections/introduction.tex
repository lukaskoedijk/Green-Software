\chapter{Introduction}
\label{ch:introduction}
% Context: what is the bigger scope of the problem you are trying to solve? Try to connect to societal/economical challenges.
% Problem Analysis: Here you present your analysis of the problem situation that your research will address.
% How does this problem manifest itself at your host organisation?
% Also summarises existing scientific insight into the problem.
%maybe not only milieu but also money
Currently more and more people are concerned with global warming. Global warming is partly the result of the emission of greenhouse gasses during energy generation of not green energy options. Because of this a lot of people want to change to green energy generation to help solve this problem. Another solution is to decrease the energy consumption. Which is not only good for the environment but can also save a lot of money on the energy bill. The energy consumption of communication networks, personal computers and data centers world wide is increasing over the years \cite{van2014trends}. This happens with a growth rate of $10\%$, $5\%$ and $4\%$ respectively \cite{van2014trends}. Therefore it is important to research ways of decreasing the energy consumption. In the field of hardware there is, according to Koomey's law, an increase of the number of computations per Joule. However this is not enough, because tasks need more computations to complete due to the confidence in the improvement of hardware \cite{verdecchia2017estimating}. For this reason we need to look at possibilities in decreasing the energy consumption from a software perspective.


\section{Problem statement}
The energy consumption of communication networks, personal computers and data centers are increasing yearly. There are two ways of decreasing the energy consumption, by decreasing the energy consumption of hardware or software. Scientific research is mostly focused on decreasing the energy consumption of hardware. There are also some papers about reducing the energy consumption in the software process and the decision making process. A small bit of research is done on the energy consumption of software, but their research goal is to estimate the energy consumption. Therefore what we miss and are going to look into is if there is a good way of writing code regards the energy consumption.

\subsection{Research questions}
%To tackle these issues, we investigate ...
During this research we answer the following two existence questions: (1) \textbf{Is there a difference in the energy consumption of software projects in different programming languages that have the same functionality} and (2) \textbf{Is there a difference in energy consumption of different software projects (in the same programming language) that have the same functionality}? To answer these questions we first need to answer the three description and classification questions listed below.
\begin{itemize}
    \item How can the energy consumption of a software project be measured?
    \item How do we proof if two programs have the same functionality?
    \item When is a difference in energy consumption big enough to be called a difference?\\
\end{itemize}

Based on the results we find when answering the second existence question, we want to look into what this difference is on code level. To do this we need to answer the descriptive/comparative question \textbf{what is the difference on code level between software project that are in the same programming language and have the same functionality, but have a difference in the energy consumption}? This question is only useful when there is a difference in the energy consumption of software projects in the same programming language that have the same functionality. \\
%Otherwise we can't link the difference in energy consumption to the difference in the code.\\

For the first research question the hypothesis is that there is a difference in the energy consumption based on the programming language chosen. The hypothesis of the second research question is that there is a difference in the energy consumption of different software projects (in the same programming language) that have the same functionality.

\subsection{Research method}
As method we use a controlled experiment. In this experiment we will run different software projects to measure the energy consumption. Here the projects are the variable input, the energy consumption the output and everything else like compiler and energy consumption calculations should be constant.

\subsubsection{Data}
For the first research question we need programs that have the same functionality, but are written in a different programming language. Possible resources for such programs are library code, interview code, student assignments and code from competitions solving math problems. We first looked at library code, but found that libraries most of the time don't use the same algorithm. This is a problem because we don't want to compare algorithms, but the way a programmer writes code. Therefor we used the \textit{the computer language benchmark game} \cite{gouy:2019} as a source for programs that have the same functionality and use the same algorithm for solving the different problems. For the second research question we need programs that have the same functionality and are written in the same language. Here we could also use \textit{the computer language benchmark game} as for most languages there are multiple programs listed.

When proving that all the projects have the same functionality we could have defined properties for the input and output and test if these properties hold for all the projects \cite{mens2004survey}, but \textit{the computer language benchmark game} already makes sure that the programs have the same functionality. Thus we don't have to do this ourselves.

To proof that the projects have different values for the energy consumption we use the one sided Mann Whitney U test. This test has as its hypothesis that two distributions are from the same population.

\subsubsection{Measuring energy consumption}
When measuring the energy consumption of a project you have to take into account the energy consumption of CPU, memory and disk \cite{acar2016impact}. The measurements can be done with a hardware or software approach. A hardware approach is more accurate but also more expensive \cite{acar2016impact}. We want to use a hardware approach and luckily as a student of the UvA we have access to the DAS-4 and the DAS-5. The DAS is a distributed system that can also do energy measurements \cite{bal2016medium}. The nodes that can measure the energy are located at the VU cluster. To measure the energy we need to run a job on the DAS-5 and use the DAS-4 to measure the energy consumption. When running a job we need to specify witch node we want to run the job on, because not all nodes can be measured for the energy consumption. 
%This is done by using the \textit{-w} flag while using the command \textit{srun}. The measuring can then be done by using the \textit{smnpwalk} command and then filtering out the information by using \textit{fgrep Energy}.

\subsubsection{Code level}
When there is a difference in the energy consumption of different software projects in the same programming language we will look at the different projects on code-level. Here we will try to find what is causing a project to have a lower or higher energy consumption. There are too many program combinations to look at, thus we need to make a selection. We choose to look at the languages that had the biggest range in energy consumption for all the problem, these languages are Python, Ruby and PHP. Then we can look for every problem to the two programs that differ the most for those three programming languages. We will look at the program combinations side by side and write down the differences that we see. Then we look at these differences and see if some are occurring more often then others. These findings will then be tested by writing our own two versions, where before testing we think one is written badly regards the energy consumption and one that is written good. We then run the two version and check if the good version has indeed a lower energy consumption using the Mann Whitney U test.

\section{Contributions}
Our research makes the following contributions:
\begin{enumerate}
	\item An energy consumption measurement set-up
	\item A data set of software programs with the same functionality
	\item Comparison of languages energy consumption
	\item Rules for writing good software regards the energy consumption
\end{enumerate}

\section{Outline}
In Chapter~\ref{ch:background} we describe the background of this thesis. 
Chapter~\ref{ch:research} describes ... 
Results are shown in Chapter~\ref{ch:results} and discussed in Chapter~\ref{ch:discussion}. Chapter~\ref{ch:related_work}, contains the work related to this thesis.
Finally, we present our concluding remarks in Chapter~\ref{ch:conclusion} together with future work.

